% pscdev3.tex		report sullo sviluppo di pscirc0.sty
% Nov 5, 1997
%%%%%%%%%%%%%%%%%%%%%%%%%%%%%%%%%%%%%%%%%%%%


\documentclass[12pt,a4paper]{articolo}
\usepackage{pscirc0}

\addtolength{\topmargin}{-1cm}
\addtolength{\textheight}{2cm}
\addtolength{\textwidth}{4cm}
\addtolength{\oddsidemargin}{-2cm}
%%%%

\title{\bfseries\vspace*{-1.5cm}
Principi e struttura del pacchetto \texttt{pscirc0.sty}}
\author{I. Maio}
\date{6 Novembre 97}
%%%%

\begin{document}
\maketitle

La definizione di un linguaggio formale per la descrizione degli schemi elettrici e la sua implementazione attraverso macroistruzioni PSTricks \`e un compito difficile.
Dopo tentativi vani, ho rinunciato alla ricerca di una soluzione ottima e ho scelto di offrire pi\`u comandi per il posizionamento degli elementi a due terminali.

Questa scelta ha tre vantaggi:
(1)~lascia all'utente la decisione sulla strategia preferita per costruire lo schema;
(2)~\`e conforme alla filosofia di PSTricks, che \`e fondato su comandi per il posizionamento di oggetti grafici generici;
(3)~separa la definizione dei simboli degli elementi circuitali da quella dei comandi usati per posizionarli.
Con questa scelta, un'unica libreria di simboli grafici \`e posizionabile attraverso nuovi criteri semplicemente costruendo nuove macroistruzioni di posizionamento.
Coerentemente alla precedenti osservazioni, ho anche privilegiato l'uso  di comandi primitivi di PSTricks (ad esempio per le etichette) e la creazione di macroistruzioni con funzioni semplici rispetto alla creazione di macroistruzioni con funzioni complesse. 

\section{Struttura del pacchetto}

\paragraph{Specificazione dei punti}
Il pacchetto sfrutta l'estensione \texttt{pst-nod} e assume che sia attiva l'opzione \texttt{SpecialCoor}.
In questo modo, i punti del piano usati per comporre il disegno si possono specificare in diversi modi:\\
(1)~mediante coordinate esplicite \texttt{(x,y)},\\
(2)~mediante nomi \texttt{(pippo)} dove \texttt{pippo} \`e definito da \verb"\pnode(x,y){pippo}",\\
(3)~mediante coordinate miste \texttt{(pippo|paperino)} e \texttt{(x,y|pippo)}.\\
Per ampliare le possibilit\`a ho aggiunto un comando per assegnare nomi a punti specificati attaverso \textbf{coordinate relative}:\\
\verb"\pnoder(x,y)(dx,dy){pluto}" assegna il nome \texttt{pluto} al punto di coordinate \verb"(x+dx,y+dy)".

\paragraph{Connessioni}
La tracciatura delle connessioni \`e completamente affidata ai comandi primitivi di PSTricks.
I comandi primitivi sono classificabili in tre tipi diversi:\\
(1)~semplici (esempio \verb"\psline..."),\\
(2)~emulatori di connessioni tra nomi (esempio \verb"\pcarc..."),\\
(3)~connessioni tra nomi (esempio \verb"\ncline...").\\
Nei comandi di tipo (1) e (2) i punti possono essere specificati con una qualunque mescola dei modi possibili, nei comandi di tipo tre i punti devono essere specificati da nomi.

Tutti i tipi di comandi sono utili, i comandi di tipo (1) permettono spezzate e curve descritte da pi\`u di due punti, quelli di tipo (2) gestiscono automaticamente parametri ausiliari che automatizzano il posizionamento di ulteriori elementi sulle connessioni e quelli di tipo (3) aggiungono ulteriori gradi di libert\`a.

\paragraph{Simboli nudi}
La parte principale del pacchetto \`e una libreria di simboli ausiliari (nodi elettrici, teste di terminali, teste di frecce) e di simboli di nuclei di elementi a due terminali (cioe` elementi a due terminali privati degli stessi).
I nuclei sono tutti disposti in posizione orizzontale con il morsetto positivo (oppure, per elementi non lineari, quello evidenziato) verso destra e sono ``opachi''.
I nomi dei comandi per la tracciatura di questi simboli hanno la struttura \verb"\ps"\textit{nome}, dove \textit{nome} identifica l'oggetto e, quando possibile, segue la sintassi \texttt{Spice}.
Esempio \verb"\psr" e \verb"\pstt" tracciano in (0,0) e orientati orizzontalmente un resistore e un elemento generico a due terminali senza i terminali.

\paragraph{Posizionamento degli elementi a due terminali}
Ho considerato tre regole diverse.
\begin{enumerate}

\item
\textbf{Posizionamento del nucleo attraverso il centro}\\
Si ottiene con \verb"\rput{ori}(pos){\ps...}" che \`e il comando di posizionamento principale di PSTricks.\\
L'aggiunta di etichette si ottiene attraverso \verb"\uput{dist}[dir](pos){etichetta}".\\
Questo via \`e utile per la costruzione di strutture particolari oppure per costruire l'intero schema tracciandone prima tutti i lati e poi sovrapponendo ad alcuni lati i simboli desiderati.
\`E una tecnica rozza ma molto efficace in alcuni casi.

\item
\textbf{Posizionamento di un elemento a due terminali dagli estremi dei terminali}\\
Si ottiene con \verb"\ttput(pos+)(pos-){\ps...}".\\
L'aggiunta di etichette si ottiene proseguendo con i comandi PSTrics \verb"_{etichetta}" (etichetta sotto) e \verb"^{etichetta}" (etichetta sopra).\\
Il comando traccia l'elemento a due terminali \verb"\ps..." disponendo l'estremo del suo terminale positivo in \verb"pos+" e quello del negativo in \verb"pos-". Per ora non sono riuscito a trovare un legge per predire la posizione finale dell'etichetta in base ai comandi ``sopra'' e ``sotto''.\\
La regola (2) \`e la pi\`u diretta ed \`e conveniente nella maggior parte dei problemi comuni.

\item
\textbf{Posizionamento di un elemento a due terminali attraverso il centro}\\
Si ottiene con \verb"\ttputc{ori}(posc){\ps...}{hnroot}{length}".\\
L'aggiunta di etichette si ottiene come nel caso (2).\\
Il comando \`e simile a \verb"\rput" ma aggiunge al nucleo i teminali in modo che l'elemento disegnato abbia lunghezza complessiva \verb"length".
Inoltre definisce i nomi dei punti occupati dagli estremi mediante
\verb"hnroot": \verb"hnroot"p \`e il nome del punto all'estremo positivo e \verb"hnroot"m quello del punto all'estremo negativo.\\
Questa regola \`e utile per la costruzione dello schema come bigrafo, disponendo prima noccioli e nodi elettrici e dopo gli archi di connessione (vedere Cap.~1 e 2 di ``Fondamenti di circuiti elettronici'', A.~Premoli).

\end{enumerate}

\paragraph{Elementi a pi\`u di due terminali}
Il loro posizionamento \`e molto pi\`u complesso e non sono riuscito a trovare regole che valessero la complessit\`a richiesta per essere implementate in PSTricks.
Per questo ho scelto la regola pi\`u facile da implementare che \`e:
un comando per ogni elemento posiziona l'elemento attraverso il centro e assegna nomi agli estremi dei suoi terminali (analogamente al comando \verb"\ttputc"). 
Con questa regola gli elementi a pi\`u di due terminali vengono trattati come noccioli e posizionati per primi.
Dopo il loro posizionamento i nomi dei punti agli estremi dei loro terminali diventano disponibili per posizionere gli altri elementi e per connetterli agli altri elementi.
%%%%%%%%%%%%%%%%%%%

\section{Esempi}

\paragraph{Posizionamento di elementi a 2 terminali}
La Figura illustra il posizionamento degli elementi a due terminali nei tre modi possibili.
L'elemento a sinistra \`e posizionato mediante \verb"\rput", quello al centro mediante \verb"\ttput" e quello a destra con \verb"\ttputc".
Gli ultimi due elementi sono etichettati e i nomi dei punti agli estremi dell'elemento di destra sono sfruttati per sovrapporre i morsetti.

\begin{verbatim}
\rput{45}(2,2){\psatt}
\ttput(5,4)(3,1){\psatt}_{$N_1$}
\ttputc{145}(6,2){\psatt}{B}{3}^{$g(v)$}
\psth(Bp)\psth(Bm)
\end{verbatim}

% simple two-port cases
\begin{center}
\begin{pspicture}(0,0)(8,4)\showgrid  
\psen
\rput{45}(2,2){\psatt}
\ttput(5,4)(3,1){\psatt}_{$N_1$}
\ttputc{145}(6,2){\psatt}{B}{3}^{$g(v)$}
\psth(Bp)\psth(Bm)
\end{pspicture}
\end{center}
%%%%

\paragraph{Definizione relativa dei punti}
La Figura illustra il posizionamento relativo dei punti per la costruzione di un ponte di Wheatstone.
Il punto F \`e definito in modo assoluto mentre tutti gli altri punti notevoli sono definiti rispetto ad F mediante \verb"\pnoder".
La parte sinistra della rete \`e costruita con spezzata e nucleo sovrapposto.

\begin{verbatim}
\pnode(7,2.5){F}\nput{0}{F}{F}
\pnoder(F)(-2,2.5){A}\nput{0}{A}{A}
\pnoder(F)(-2,-2.5){B}\nput{0}{B}{B}
\pnoder(F)(-4,0){C}\nput{0}{C}{C}
\pnoder(F)(-5,0){D}\nput{0}{D}{D}
%
\ttput(F)(A){\psatt}\ttput(F)(B){\pstt}
\ttput(C)(A){\pstt}\ttput(C)(B){\pstt}
%
\psline(A)(D|A)(D|B)(B)\rput{90}(D){\pstt}
\end{verbatim}

% relative nodes
\begin{center}
\begin{pspicture}(0,0)(8,5) 
\psen
\pnode(7,2.5){F}\nput{0}{F}{F}
\pnoder(F)(-2,2.5){A}\nput{0}{A}{A}
\pnoder(F)(-2,-2.5){B}\nput{0}{B}{B}
\pnoder(F)(-4,0){C}\nput{0}{C}{C}
\pnoder(F)(-5,0){D}\nput{0}{D}{D}
%
\ttput(F)(A){\psatt}\ttput(F)(B){\pstt}
\ttput(C)(A){\pstt}\ttput(C)(B){\pstt}
%
\psline(A)(D|A)(D|B)(B)\rput{90}(D){\pstt}
\end{pspicture}
\end{center}
%%%%

\paragraph{Schemi informali}
L'uso della descrizione mediante bigrafi si presta alla creazione di schemi inusuali con collegamenti curvilinei.
La Figura esemplifica l'uso dei comandi \verb"\ttputc" e \verb"\pcarc" per la creazione di un parallelo di due elementi con connessioni curvilinee.

\begin{verbatim}
\ttputc{0}(3,3){\psatt}{B1}{0}^{$N_1(v)$}
\ttputc{0}(3,1){\pstt}{B2}{0}
\pcarc[arcangle=20](B1p)(6,2)\pcarc[arcangle=20](6,2)(B2p)
\pcarc[arcangle=20](B2m)(0,2)\pcarc[arcangle=20](0,2)(B1m)
\psn(6,2)\psn(0,2)
\end{verbatim}

% \ttputc check
\begin{center}
\begin{pspicture}(0,0)(6,4)  
\psen
\ttputc{0}(3,3){\psatt}{B1}{0}^{$N_1(v)$}
\ttputc{0}(3,1){\pstt}{B2}{0}
\pcarc[arcangle=20](B1p)(6,2)\pcarc[arcangle=20](6,2)(B2p)
\pcarc[arcangle=20](B2m)(0,2)\pcarc[arcangle=20](0,2)(B1m)
\psn(6,2)\psn(0,2)
\end{pspicture}
\end{center}
%%%%

\paragraph{Posizionamento di elementi con pi\`u di due terminali}
La Figura illustra l'uso del pomando \verb"\psccoa" per disegnare un amplificatore operazionale disegnato in verso cc (counter-clockwise, cio\`e invertente, non invertente, uscita) con orientamento arbitrario.
I nomi dei punti estremi dei terminali sono creati con la regola \texttt{hmroot}m, \texttt{hmroot}p, \texttt{hmroot}o, \texttt{hmroot}g e sono associati nell'ordine ai morsetti $-$, $+$, uscita e terra.
I nomi dei punti estremi sono poi usati per tracciare la connessione di fantasia che compare mediante un comando \verb"\psline".

\begin{verbatim}
\psccoa{10}(4,2){a1}{4}
\pnoder(a1m)(-3,-2){P}
\psline(a1m)(P|a1m)(P)(a1g|P)(a1g)
\rput(a1g|P){\psgr}
\end{verbatim}

% oa check
\begin{center}
\begin{pspicture}(0,0)(8,4)\showgrid  
\psen
\psccoa{10}(4,2){a1}{4}
\pnoder(a1m)(-3,-2){P}
\psline(a1m)(P|a1m)(P)(a1g|P)(a1g)
\rput(a1g|P){\psgr}
\end{pspicture}
\end{center}
%%%%

\paragraph{Esempio finale}
La Figura illustra un esempio complesso con due elementi con pi\`u di due terminali.
Per primo viene posizionato l'amplificatore e mediante i nomi dei suoi punti estremi vengono definiti gli altri punti notevoli dello schema.
Gli elementi rimanenti sono aggiunti usando una miscela delle tecniche descritte.

\begin{verbatim}
\pnode(7,2){C}
\psccoa{0}(C){a}{4}
\pnoder(am)(-3,-1){tpc}\pnoder(am)(5,-.5){oa}
\pnoder(am)(5,-2){ob}\pnoder(C)(0,1.5){F}
\pstp{0}(tpc){tp}{2}{2}
%
\ttput(am)(tp21){\psatt}_{Z(s)}
\ttput(tp11)(tp12){\pstt}
\psline(am)(am|F)(ao|F)(ao)(oa)\rput(F){\pstt}
%
\psline(tp22)(ob)\ttput(ap)(ap|ob){\psah}^{$i_+$}
\psline(ag)(ag|ob)
\psth(oa)\psth(ob)\rput(ag|ob){\psgr}
%
\psvrd(oa)(ob){-30}{$v_u(t)$}
\end{verbatim}

% final complex example
\begin{center}
\begin{pspicture}(0,0)(9,5) 
\psen
\pnode(7,2){C}
\psccoa{0}(C){a}{4}
\pnoder(am)(-3,-1){tpc}\pnoder(am)(5,-.5){oa}
\pnoder(am)(5,-2){ob}\pnoder(C)(0,1.5){F}
\pstp{0}(tpc){tp}{2}{2}
%
\ttput(am)(tp21){\psatt}_{Z(s)}
\ttput(tp11)(tp12){\pstt}
\psline(am)(am|F)(ao|F)(ao)(oa)\rput(F){\pstt}
%
\psline(tp22)(ob)\ttput(ap)(ap|ob){\psah}^{$i_+$}
\psline(ag)(ag|ob)
\psth(oa)\psth(ob)\rput(ag|ob){\psgr}
%
\psvrd(oa)(ob){-30}{$v_u(t)$}
\end{pspicture}
\end{center}


\end{document}
