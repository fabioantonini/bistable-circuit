% pstype.tex		prototypes of pscirc commands
%
% A.Premoli and I.Maio
% Sep 4, 1998
%%%%%%%%%%%%%%%%%%%%%%%%%%%%%%%%%%%%%%%%%%%%


\documentclass[12pt,a4paper]{article}
%\usepackage{pscirc3}
\usepackage{pst-circ}
\addtolength{\textwidth}{3cm}
\addtolength{\oddsidemargin}{-1.5cm}


\title{\bfseries\vspace*{-1.5cm}
\texttt{pstype.tex}: prototypes of pscirc commands}
\author{A. Premoli and I. Maio}
\date{29 September 98}
%%%%

\begin{document}
\maketitle
%%%%%%%%%%%%%%%%%%%%%%%%%%%%%%%%%%%%%

\subsection*{Command types}
\begin{itemize}
\item[(1)] headings
\item[(2)] environment
\item[(3)] pstrick nodes
\item[(4)] auxiliary elements
\item[(5)] 2-terminal elements
\item[(6)] reference directions
\item[(7)] multiport elements
\end{itemize}
%%%%%%%%%%%%%%%%%%%%%%%%%%%%%%%%%%%%%

\begin{verbatim}

1 headings			%%%%%%%%%%%%%%%%%%%%%%%%%%%%%%%%%%%%%%%%%%%%
\documentclass[12pt,a4paper]{article}
\usepackage{pscirc3}

\pnode(){n1}\pnode(){n2}\pnode(){n3}\pnode(){n4}\pnode(){n5}\pnode(){n6}\pnode(){n7}\pnode(){n8}\pnode(){n9}\pnode(){n10}\pnode(){n11}\pnode(){n12}\pnode(){n13}\pnode(){n14}\pnode(){n15}\pnode(){n16}\pnode(){n17}\pnode(){n18

2 environment		%%%%%%%%%%%%%%%%%%%%%%%%%%%%%%%%%%%%%%%%%%%%
%% \psset modifies default units of x and y, num_length is a real number
%% with units (eg; pt, mm, cm, ...), if units are dropped cm are assumed
%% \psgrids draws a reference grid
%
%\psset{unit=	num_length}
\begin{pspicture}(x size, y size)
\psen
%\psgrid

... commands drawing the circuit ...

\end{pspicture}


3 pstrick nodes		%%%%%%%%%%%%%%%%%%%%%%%%%%%%%%%%%%%%%%%%%%%
%% single node defined by absolute coordinate
\pnode(x,y){node_name}

%% single node defined by shifted coordinates
%% ie, node_name is located at
%% (x_reference_node_name+dx,y_reference_node_name+dy)
\pnoder(reference_node_name)(dx,dy){node_name}

%% grid of nodes, llc=lower left corner of the grid frame
\psng(x_llc_offset,y_llc_offset)(x_step,y_step){no_x_nodes}{no_y_nodes}


4 auxiliary elements	%%%%%%%%%%%%%%%%%%%%%%%%%%%%%%%%%%%%%%%%%
%% node
\psn(x,y)
\psn(node_name)

%% terminal head
\psth(x,y)
\psth(node_name)

%% arrow head, plus and minus symbols (to be placed by pst commands)
\psah
\plu
\mnu

%% white spot to denote conductor crossings
\psdot[dotsize=.25,linecolor=white](x,y)
\psdot[dotsize=.25,linecolor=white](node_name)

%% one-terminal with ground
\psgr(x,y)
\psgr(node_name)

%% basic placement and labelling commands ...
%% relative placement
\uput{label_distance}[label_angle(u,r,l,... or_degree)](x,y){label}
\uput{label_distance}[label_angle(u,r,l,... or_degree)](node_name){label}
%% absolute placement
\rput{rotation_angle}(x,y){stuff}
\rput{rotation_angle}(node_name){stuff}

%% piecewise linear curve connecting several points, point coordinates
%% can be specified by different formats
\psline[linestyle=(dashed,dotted)]%
(x1,y1)(node_name2)(node_nameA|node_nameB)...

%% line to connect two nodes and to bear a current reference direction
%% (\pssa)
\ttput(plus_node_name)(minus_node_name){}
\pssa{arrow_dist}{arrow_way(0=loads,180=sources)}
{label_distance}{label_angle(u,r,l,... or_degree)}{label}

%% curve connecting several points, point coordinates
%% can be specified by different formats
%% the arrow argument produce an arrow head at the hend of the curve
\pscurve[linestyle=(dashed,dotted)]{->}%
(x1,y1)(node_name2)(node_nameA|node_nameB)...

%% arc to connect two nodes and to bear a current reference direction
%% (\pssa)
\pcarc[arcangle= ](minus_node_name)(plus_node_name)%
\pssa{arrow_dist}{arrow_way(0=loads,180=sources)}
{label_distance}{label_angle(u,r,l,... or_degree)}{label}


5 two-terminal elements %%%%%%%%%%%%%%%%%%%%%%%%%%%%%%%%%%%%%%
%% 2-terminal element and its label,
%% label_distance>0 and label_angle is referred to the positive x axis
\ttput(plus_node_name)(minus_node_name){element_name}
\ncput{\uput{label_distance}[label_angle(u,r,l,... or_degree)](0,0){label}}

6 reference directions %%%%%%%%%%%%%%%%%%%%%%%%%%%%%%%%%%%%%%%
%% free arrow (to draw the reference direction of the voltage between
%% a node pair and the reference direction of current entering/leaving
%% a terminal)
%% if arc_param > 0 the arrow is curved like a right hand corner
%% label_distance>0 and label_angle is referred to the positive x axis
\psfa(plus_node_name)(minus_node_name){arc_param}
{label_distance}{label_angle(u,r,l,... or_degree)}{label}

%% superimposed arrow (to draw a current reference direction on a terminal)
%% the command is to be appended to the command placing the element
%% whose terminal carries the refenced current,
%% arrow_distance shifts the harrow head of arrow_distance from the element
%% center toward the positive node,
%% label_distance>0, and label_angle is referred to the positive x axis
\pssa{arrow_dist}{arrow_way(0=orientation according to associate reference directions,180= the opposite orientation)}
{label_distance}{label_angle(u,r,l,... or_degree)}{label}


7 multiport elements %%%%%%%%%%%%%%%%%%%%%%%%%%%%%%%%%%%%%%%%%
%% counter-clockwise OA
%% node names at terminal heads are root_node_names(m,p,o,g)
\psccoa{angle}(center_node){root_node_names}{overall_length}

%% clockwise OA
%% node names at terminal heads are root_node_names(m,p,o,g)
\psen\pscwoa{angle}(center_node){root_node_names}{overall_length}

%% two-port element
%% node names at terminal heads are root_node_names(11,12) left
%% and root_node_names(21,22) right
\pstp{angle}(center_node){root_node_names}{overall_length}{port_width}

%% nullor
%% node names at terminal heads are root_node_names(11,12) left
%% and root_node_names(21,22) right
\psnn{angle}(center_node){root_node_names}{overall_length}{port_width}

%% three-terminal element
%% node names at terminal heads are root_node_names(1,2,g)
\psttt{angle}(center_node){root_node_names}{overall_length}
{terminal_height}

%% three-terminal nullor
%% node names at terminal heads are root_node_names(1,2,g)
\psnnt{angle}(center_node){root_node_names}{overall_length}
{terminal_height}

%% ideal transformer
%% node names at terminal heads are root_node_names(11,12) left
%% and root_node_names(21,22) right
\psit{angle}(center_node){root_node_names}{port_width}

%% ideal gyrator
%% node names at terminal heads are root_node_names(11,12) left
%% and root_node_names(21,22) right
\psig{angle}(center_node){root_node_names}{port_width}

%% coupled inductors
%% node names at terminal heads are root_node_names(11,12) left
%% and root_node_names(21,22) right
\psll{angle}(center_node){root_node_names}{port_width}

\end{verbatim}

\end{document}
