\documentclass[a4paper,12pt]{article}
\usepackage{color}
\usepackage{graphicx}
\usepackage{float}
\usepackage{amsfonts}
\usepackage{siunitx}
\usepackage{amsmath}

\begin{document}

\title{Elaborato di Matematica e Fisica}
\author{Mattia Antonini \\ Liceo Scientifico Marco Vitruvio Pollione\\Classe 5R}
\date{\today}
\maketitle

\pagenumbering{roman}
\tableofcontents
\newpage
\pagenumbering{arabic}

\section{Introduzione}
Traccia dell’elaborato concernente le discipline di indirizzo individuate come
oggetto della seconda prova scritta ai sensi dell’O.M. n.10 del 16 maggio 2020.

\begin{center}
Classe 5 Sez.R
\end{center}
\section{Problema}
Tra le infinite funzioni del tipo $y = ae^x + be^{-x}$

\begin{enumerate}
\item Determinare la $f_1(x)$  in corrispondenza della quale la curva rappresentativa $C_1$ ammette un minimo in $(0,4)$ e la $f_2(x)$ in corrispondenza della quale la curva rappresentativa $C_2$ passi per l’origine del sistema di riferimento e sia ivi tangente alla retta di equazione $y = 4x$.

\item Disegnare le due curve evidenziandone le simmetrie. Si provi, inoltre che è sempre verificata la relazione $f_1(x) > f_2(x)$. 

\item  Calcolare l’area della regione di piano delimitata dalle due curve, dall’asse delle $y$ e dalla retta $x = k\ (k>0)$ ed il limite a cui tende la suddetta area al tendere di $k \rightarrow +\infty$.
\item In un riferimento cartesiano, dove le lunghezze sono espresse in metri $(m)$, le due curve, l’asse delle $y$ e la retta $x =1$ rappresentino il contorno di una spira immersa in un campo magnetico perpendicolare ad essa e variabile secondo la legge $$B = B_0e^{-\pi t}cos(\pi t)$$ 
con $B_0 = 2.0\cdot10^{-2}\ T$ e $ t >0$.
Supponendo che la spira abbia una resistenza pari a $10\ \Omega$ determinare l’intensità di corrente che circola nella spira. Spiegare quale relazione esiste tra la variazione del campo che induce la corrente e il verso della corrente indotta.

\end{enumerate}
\newpage
\section{Quesiti}
\begin{enumerate}
\item Data la funzione
\begin{equation} f(x)=
\left\{
\begin{array}{l}
 -3x^2+Hx\;\; x \le 1\\ \frac{K}{x^2}\;\; x>1
\end{array}
\right.
\end{equation}

Determinare $H$ e $K$ in modo che la funzione risulti continua e derivabile in $x=1$.

\item  Sia $f(x)$ una funzione reale di variabile reale, continua nel campo reale, tale che $f(0) = 2$. calcolare: $$\lim_{x\rightarrow 0}\frac{\int_{0}^x f(t)dt}{2xe^x}$$
\item Un protone in moto con velocità pari a $5\cdot10^6\ m/s$ si muove in un campo magnetico uniforme uscente dalla pagina e di modulo $3\cdot10^{-3}\ T$; qual è il raggio di curvatura della traiettoria descritta dal protone?
\item Sia dato il circuito in figura, con $R_1 = 5\ \Omega, R_2 = 4\ \Omega, R_3 = 10\ \Omega, R_4 = 50\ \Omega,  R_5 = 6\ \Omega$.

\begin{figure}[H]
\centering
\includegraphics[width=10cm]{circuito.png}
\caption {Circuito}
\end{figure}
Al circuito è applicata una differenza di potenziale di $75 V$.  Calcolare:
\begin{enumerate}
\item la resistenza equivalente;
\item la corrente $i_3$ che passa attraverso la resistenza $R_3$;
\item la caduta di potenziale ai capi di ogni resistenza.
\end{enumerate}
\end{enumerate}

\newpage
\section{Soluzioni del Problema}

\begin{enumerate}
\item  Determinare la $f_1(x)$  in corrispondenza della quale la curva rappresentativa $C_1$ ammette un minimo in $(0,4)$ e la $f_2(x)$ in corrispondenza della quale la curva rappresentativa $C_2$ passi per l’origine del sistema di rifermento e sia ivi tangente alla retta di equazione $y = 4x$.

Sia definita la funzione \begin{equation}y =f_1(x) = ae^x+ be^{-x}\end{equation} 
Si imponga che la sua derivata prima abbia  il minimo nel punto di coordinate $(0,4)$
\begin{equation}\label{cond1} \frac{df_1(x)}{dx}=0\end{equation} 
Si può anche assumere che funzione $f_1(x)$ passerà per il punto $(0,4)$
\begin{equation}\label{cond2}f_1(0)=4\end{equation}
Calcoliamo la derivata prima della funzione $f_1(x)$:
\begin{equation}\label{derf1} \frac{df_1(x)}{dx} = \frac{d(ae^x)}{dx} + \frac{d(be^{-x})}{dx} = ae^x-be^{-x}\end{equation} 
Si ponga a $0$ la derivata prima per identificare per quali valori della variabile $x$ essa si annulla:
\begin{equation} ae^x-be^{-x}=0\end{equation}
\begin{equation} ae^x=be^{-x}\end{equation}
Dopo alcuni semplici passaggi si ottiene:
\begin{equation}\ln{a}+x = \ln{b}-x\end{equation}
\begin{equation}2x = \ln{b}-\ln{a}\end{equation}
\begin{equation}x =\frac{1}{2}(\ln{b}-\ln{a})\end{equation}
Imponendo che questa ascissa sia pari a $0$ si ottiene:
\begin{equation}\frac{1}{2} (\ln{b}-\ln{a}) = 0\end{equation}
\begin{equation}\ln\frac{b}{a}= 0\end{equation}
Per la proprietà dei logaritmi si ha:
\begin{equation}\frac{b}{a}= 1\end{equation}
Si ricava quindi una prima relazione tra i due parametri $a$ e $b$:
\begin{equation}\label{cond6}a=b\end{equation}
Utilizziamo la relazione (\ref{cond2}) per ricavare una seconda relazione tra i due parametri $a$ e $b$:
\begin{equation}4=ae^0+b^0\end{equation}
In base alla relazione (\ref{cond6}) sappiamo che $a=b$ e quindi possiamo scrivere:
\begin{equation}2b=4\end{equation}
\begin{equation}b=2\end{equation}
e quindi:
\begin{equation}a=2\end{equation}
Dunque la funzione $f_1(x)$ può essere scritta come:
\begin{equation}C_1: y =f_1(x) = 2e^x+2e^{-x}\end{equation}
Passiamo a determinare l'equazione $C_2:f_2(x)$ imponendo le condizioni di passaggio per l'origine $(0,0)$ e che la funzione sia ivi tangente alla retta di equazione $y=4x$.
%
\begin{subequations}\label{cond3}
\begin{eqnarray}
f_2(0)=0\\
ae^0+be^{-0}=0\\
a=-b
\end{eqnarray}
\end{subequations}
%
Usiamo la condizione di tangenza nell'origine per ricavare una seconda relazione tra i parametri $a$ e $b$. La retta $y=4x$ ha coefficiente angolare pari a $4$. Imponiamo che la derivata prima della funzione $f_2(x)$ nell'origine sia pari a $4$.
\begin{equation}\frac{df_2(x)}{dx}=ae^x-be^{-x}\end{equation}
Usando la relazione $a=-b$ si ottiene:
\begin{equation}\frac{df_2(x)}{dx}=-be^x-be^{-x}\end{equation}
Imponiamo che tale derivata sia pari a $4$ per $x=0$:
\begin{equation}-2b=4\end{equation}
\begin{equation}b=-2\end{equation}
\begin{equation}a=2\end{equation}
\begin{equation}C_2: y=f_2(x)=2e^x-2e^{-x}\end{equation}

\item Disegnare le due curve evidenziandone le simmetrie. Si provi, inoltre che è sempre verificata la relazione $f_1(x) > f_2(x)$.
\begin{equation}y=f_1(x)=2e^x+2e^{-x}\end{equation}
Tale funzione è definita per qualunque $x$. Quindi possiamo dire che il suo dominio $D$ è $\forall x\; \epsilon\; \mathbb{R}$.

Si studi ora il segno imponendo $y>0$.
\begin{equation}\label{cond4}2e^x+2e^{-x}>0\end{equation}
Poichè le funzioni esponenziali $e^x$ e $e^{-x}$ sono sempre positive se ne deduce che anche la funzione $f_1(x)$ è sempre $>0$.
\begin{equation}y=f_1(x)=2e^x+2e^{-x}>0\;\; \forall x\; \epsilon\; \mathbb{R}\end{equation}
Si calcoli l'intersezione della $y=f_1(x)=2e^x+2e^{-x}$ con l'asse delle $y$ imponendo $x=0$:
\begin{equation}f_1(0)=2e^0+2e^{-0}=4\;\;\end{equation}
L'intersezione con l'asse delle $y$ risulta $ Int(y): (0,4)$ che coincide con il minimo trovato precedentemente.

Come si è dimostrato nella relazione (\ref{cond4}) la funzione $f_1(x)$ è sempre positiva e quindi non c'è alcuna intersezione con l'asse delle $x$.

Passiamo ad analizzare le simmetrie.

Una funzione è pari se $f(x)=f(-x)$. Imponendo questa condizione alla $f_1(x)$ per $x=1$ e $x=-1$ si verifica che:
\begin{equation}f_1(1)=2e^1+2e^{-1}=2(e+\frac{1}{e})\end{equation}
\begin{equation}f_1(-1)=2e^{-1}+2e^1=2(e+\frac{1}{e})\end{equation}
Dunque risulta che 
\begin{equation}f_1(-1)=f_1(1)\end{equation}
La funzione $f_1(x)$ risulta dunque pari.

Si calcolino ora gli asintoti orizzontali sia per $x \rightarrow\;\infty$ che per $x \rightarrow\;-\infty$.
\begin{equation}
  \lim_{x\rightarrow \infty}(2e^x+2e^{-x})=2e^{+\infty}+\frac{2}{e^{+\infty}}=+\infty
\end{equation}
\begin{equation}
  \lim_{x\rightarrow- \infty}(2e^x+2e^{-x})=\frac{2}{e^{+\infty}}+2e^{+\infty}=+\infty
\end{equation}
La derivata prima di $f_1(x)$ si può calcolare come:
\begin{equation}\frac{df_1(x)}{dx}=2e^x-2e^{-x}\end{equation}
Ponendo $\frac{df_1(x)}{dx}=0$ si ottiene
\begin{equation}2e^x-2e^{-x}=0\end{equation}
Da cui si ricava che la derivata prima si annulla per $x=0$.
Studiando il segno della derivata prima si ottiene:
\begin{equation}2e^x-2e^{-x}>0\end{equation}
e quindi
\begin{equation}e^x> e^{-x}\end{equation}
da cui
\begin{equation}x>0\end{equation}
\begin{figure}[H]
\centering
\includegraphics[width=1\textwidth]{derivata-f1.jpg}
\caption {Studio del segno della derivata di $f_1(x)$}
\end{figure}
La funzione $f_1(x)$ calcolata per $x=0$ vale $f_1(0)=4$.
La funzione $f_1(x)$ ha quindi un minimo nel punto di coordinate $(0,4)$. Questo risultato è una logica conseguenza  di come i parametri $a$ e $b$ sono stati determinati. 
Si valuti ora la derivata seconda della funzione $f_1(x)$:
\begin{equation} \frac{d^2f_1(x)}{dx^2}=2e^x+2e^{-x}\end{equation}
Tale funzione non si azzera per alcun valore di $x$. Non ci sono quindi flessi.

Il grafico della funzione $f_1(x)$ è illustrato nella seguente figura:

\begin{figure}[H]
\centering
\includegraphics[width=1\textwidth]{f1x.png}
\caption {$f_1(x)=2e^x+2e^{-x}$}
\label{$f_1(x)$}
\end{figure}

Passiamo  a graficare la funzione $f_2(x)$.
\begin{equation}y=f_2(x)=2e^x-2e^{-x}\end{equation}
Il dominio $D$ della funzione $f_2(x)$ è:  $\forall x\; \epsilon\; \mathbb{R}$. Non ci sono dunque asintoti verticali.

Studiamone ora il segno:
\begin{equation}2e^x-2e^{-x}>0\end{equation}
 da cui si ricava:
\begin{equation}x>-x\end{equation}
\begin{equation}x>0\end{equation}
Quindi la funzione inverte il suo segno nell'origine $(0,0)$.
L'intersezione con l'asse $y$ si ottiene imponendo $x=0$ da cui:
\begin{equation}y=2e^0-2e^{0}=0\end{equation}
Verifichiamo se la funzione sia pari o dispari calcolandone il valore per $x=1$ e $x=-1$:
\begin{equation}f_2(1)=2e^1-2e^{-1}\end{equation}
\begin{equation}f_2(-1)=2e^{-1}-2e^1\end{equation}
Dato che i due valori sono diversi se ne deduce che la funzione non è pari.
Tuttavia si verifica facilmente che $f_2(-1)=-f_2(1)$. La funzione dunque è dispari.

Si calcolino ora gli asintoti orizzontali sia per $x \rightarrow\;\infty$ che per $x \rightarrow\;-\infty$.
\begin{equation}
  \lim_{x\rightarrow \infty}(2e^x-2e^{-x})=2e^{\infty}-\frac{2}{e^{+\infty}}=+\infty
\end{equation}
\begin{equation}
  \lim_{x\rightarrow- \infty}(2e^x+2e^{-x})=\frac{2}{e^{+\infty}}-2e^{+\infty}=-\infty
\end{equation}

La derivata prima della funzione $f_2(x)$ risulta:
\begin{equation}\frac{df_2(x)}{dx}=2e^x+2e^{-x}\end{equation}
da cui, imponendo l'azzeramento della stessa, si ricava con semplici passaggi che:
\begin{equation}e^x+\frac{1}{e^x}=0\end{equation}
Tale equazione non ammette radici e pertanto possiamo concludere che la funzione $f_2(x)$ non presenta nè massimi nè minimi relativi.
Poichè la derivata prima è sempre positiva ne deduciamo che la funzione è crescente.

Andiamo  a valutare ora la derivata seconda per ricercare eventuali flessi.
\begin{equation}\frac{d^2f_2(x)}{dx^2}=2e^x-2e^{-x}\end{equation}
Imponendo che 
\begin{equation}\frac{d^2f_2(x)}{dx^2}=2e^x-2e^{-x}>0\end{equation}
si ricava che
\begin{equation}2e^x>2e^{-x}\end{equation}
che equivale a:
\begin{equation}x>-x\end{equation}
e quindi
\begin{equation}x>0\end{equation}
Dunque la funzione $f_2(x)$ ha un flesso nell'origine degli assi cartesiani.

\begin{figure}[H]
\centering
\includegraphics[width=1\textwidth]{concavità.jpg}
\caption {Studio concavità della funzione $f_2(x)$}
\end{figure}

La funzione $f_2(x)$ è rappresentata nel seguente grafico.
\begin{figure}[H]
\centering
\includegraphics[width=1\textwidth]{f2x.png}
\caption {$f_2(x)=2e^x-2e^{-x}$}
\end{figure}

Si verifichi ora che 
\begin{equation}f_1(x)>f_2(x)\end{equation}

\begin{equation}2e^x+2e^{-x}>2e^x-2e^{-x}\end{equation}
\begin{equation}4e^{-x}>0\;\;e^{-x}>0\end{equation}
Questa relazione è verificata  $\forall x\; \epsilon\; \mathbb{R}$
Dunque $f_1(x) > f_2(x)$  $\forall x\; \epsilon\; \mathbb{R}$.

\item  Calcolare l’area della regione di piano delimitata dalle due curve, dall’asse delle $y$ e dalla retta $x = k$ $(k>0)$ ed il limite a cui tende la suddetta area al tendere di $k \rightarrow \infty$.

Per calcolare l'area della regione di piano delimitata dalle due curve, dall’asse delle $y$ e dalla retta $x = k$ $(k>0)$ riportiamo qui di seguito le equazioni delle funzioni che delimitano la superficie di cui si chiede di calcolare l'area:
\begin{equation}y=f_1(x)=2e^x+2e^{-x}\end{equation}
\begin{equation}y=f_2(x)=2e^x-2e^{-x}\end{equation}
\begin{equation}x=0\end{equation}
\begin{equation}x=k\;\; k>0\end{equation}
Le due funzioni $f_1(x)$ e $f_2(x)$ non si intersecano mai anche se la loro differenza tende a decrescere all'aumentare della $x$.
L'area può essere espressa come 
\begin{equation}A=\int_{0}^k (f_1(x)-f_2(x))dx\end{equation}
\begin{equation}A=\int_{0}^k (4e^{-x})dx=-4[e^{-x}]_{0}^k = 4-4e^{-k}\end{equation}
Dunque possiamo calcolare il limite per $k\rightarrow +\infty$
\begin{equation} \lim_{k\rightarrow +\infty} (4-4e^{-k})=4\end{equation}

\begin{figure}[H]
\centering
\includegraphics[width=1\textwidth]{f1_f2.png}
\caption {$f_1(x), f_2(x), x=0, x=k$}
\end{figure}

\item In un riferimento cartesiano, dove le lunghezze sono espresse in metri $(m)$, le due curve, l’asse delle $y$ e la retta $x =1$ rappresentino il contorno di una spira immersa in un campo magnetico perpendicolare ad essa e variabile secondo la legge $$B = B_0e^{-\pi t}cos(\pi t)$$ 
con $B_0 = 2.0\cdot10^{-2} T$ e $ t >0$.
Supponendo che la spira abbia una resistenza pari a $10\ \Omega$ determinare l’intensità di corrente che circola nella spira. Spiegare quale relazione esiste tra la variazione del campo che induce la corrente e il verso della corrente indotta.

Procediamo con il calcolo della f.e.m. indotta dal campo magnetico
\begin{equation} B(t)=B_0e^{-\pi t}cos(\pi t) \;\; B_0=2\cdot10^{-2} \;T\end{equation}
nella spira delimitata da $f_1(x), f_2(x), x=0, x=1$.

La resistenza della spira è pari a $R=10\ \Omega$.

Dunque in base alla legge di Ohm possiamo affermare che 
\begin{equation}f.e.m.=IR\end{equation}
e dunque
\begin{equation}I=\frac{f.e.m.}{R}\end{equation}


In base alla legge di Faraday sappiamo che la f.e.m. indotta in un circuito da un campo magnetico variabile nel tempo è pari a:
\begin{equation}f.e.m.=-\frac{d\Phi_B(t)}{dt}\end{equation}
Sapendo che il il campo magnetico è uniforme su tutta la superficie $S$ possiamo calcolare il flusso $\Phi_B(t)$:
\begin{equation}\Phi_B(t)=SB(t)cos(\alpha)\end{equation}
con $\alpha=0$ nel nostro caso in quanto il campo magnetico è ortogonale alla superficie della spira. Quindi $cos(0)=1$.
\begin{figure}[H]
\centering
\includegraphics[width=1\textwidth]{f1_f2_b.jpg}
\caption {$f_1(x), f_2(x), x=0, x=1$}
\end{figure}
Ricordiamo che $S=4-4e^{-k}$
Nel caso in esame $k=1$, dunque
\begin{equation} S=4-4e^{-1} \simeq 2.52\  m^2\end{equation}
\begin{equation}f.e.m.=-SB_0\frac{d(e^{-\pi t}cos(\pi t))}{dt}\end{equation}
\begin{equation}\frac{d(e^{-\pi t}cos(\pi t))}{dt}=cos(\pi t)\frac{d(e^{-\pi t})}{dt}+e^{-\pi t}\frac{d(cos(\pi t))}{dt}\end{equation}
\begin{equation}=-\pi cos(\pi t) e^{-\pi t}-\pi e^{-\pi t}sin(\pi t)\end{equation}
\begin{equation}=-\pi e^{-\pi t}( cos(\pi t) + sin(\pi t))\end{equation}
Quindi la $f.e.m.$ risulta:
\begin{equation}f.e.m.=SB_0[\pi e^{-\pi t}(cos(\pi t)+ sin(\pi t)]\end{equation}
Si può osservare che la $f.e.m.$ indotta è periodica, ma la sua ampiezza si va riducendo con il tempo in maniera esponenziale.
Ciò è conseguenza del fatto che il campo magnetico (e quindi anche il suo flusso attraverso la superficie $S$) ha anche esso un'ampiezza che si va riducendo in maniera esponenziale con il tempo.

Dunque la corrente indotta nel circuito di resistenza $R$ risulta essere:

\begin{equation}I=\frac{SB_0[\pi e^{-\pi t}(cos(\pi t)+ sin(\pi t)]}{R}\ A\end{equation}
\begin{equation}\;=5\cdot10^{-3}\;\pi e^{-\pi t}(cos(\pi t)+ sin(\pi t))\ A\end{equation}

L'andamento temporale della $f.e.m.$ è illustrato nella seguente figura:

\begin{figure}[H]
\centering
\includegraphics[width=1\textwidth]{fem.png}
\caption {$f.e.m.$}
\end{figure}

Assumendo per convenzione che:
\begin{enumerate}
\item il campo magnetico sia positivo quando esce dal foglio e negativo quando vi entra (è variabile nel tempo)
\item il vettore associato alla superficie sia orientato uscente dal foglio
\end{enumerate}
applicando la regola della mano destra possiamo affermare che
\begin{enumerate}
\item quando la derivata del campo magnetico e  quindi del flusso concatenato e' positiva, la $f.e.m.$ e' negativa e quindi fa circolare la corrente in modo che produca un flusso opposto (negativo in questo caso) a quello concatenato e la corrente scorre in senso orario
\item quando la derivata del campo magnetico e  quindi del flusso concatenato e' negativa, la $f.e.m.$ e' positiva e quindi fa circolare la corrente in modo che produca un flusso concorde (positivo in questo caso) a quello concatenato e la corrente scorre in senso antiorario
\end{enumerate}
In sostanza la $f.e.m.$ indotta dalla variazione del flusso $\Phi_B(t)$ attraverso la superficie $S$ produce una corrente che si oppone alla variazione del campo magnetico concatenato.
\end{enumerate}

\section{Soluzioni dei Quesiti}

\begin{enumerate}
\item \textbf{Quesito 1}

Si determinino $H$ e $K$ affinchè la seguente funzione sia continua e derivabile nel punto di accumulazione di ascissa $x=1$.
\begin{equation} f(x)=
\left\{
\begin{array}{l}
 -3x^2+Hx\;\; x \le 1\\ \frac{K}{x^2}\;\; x>1
\end{array}
\right.
\end{equation}
Una funzione si dice continua in un suo punto di accumulazione $x_0$ se 
\begin{enumerate}
\item i due limiti destro e sinistro esistono finiti ed hanno lo stesso valore
\item il comune valore dei limiti destro e sinistro coincide con la valutazione della funzione nel punto
\end{enumerate}
\begin{equation}\lim_{x\rightarrow x_0^{+}} f(x) = \lim_{x\rightarrow x_0^{-}} f(x) = f(x_0)\end{equation}

Inoltre una funzione si dice derivabile in un punto se esistono finiti e coincidono il limite destro e sinistro del rapporto incrementale calcolato nel punto.
\begin{equation}\lim_{h\rightarrow 0^{+}} \frac{f(x_0+h) - f(x_0)}{h} = \lim_{h\rightarrow 0^{-}} \frac{f(x_0+h) - f(x_0)}{h}\end{equation}

Applicando queste due condizioni all'equazione $f(x)$ nel punto di accumulazione $x_0=1$ sarà possibile identificare i valori di H e K.
Eguagliamo i due limiti per $x \rightarrow 1$ sia da destra e da sinistra. Essi coincidono con il valore della funzione calcolata per $x=1$.
\begin{equation}\lim_{x\rightarrow 1^{+}} f(x) = \lim_{x\rightarrow 1^{-}} f(x)=f(1)\end{equation}
Si ottiene:
\begin{equation} -3+H=K\end{equation}

Per ottenere una seconda relazione sfruttiamo la condizione di derivabilità imponenendo l'uguaglianza della derivata calcolata per $x\rightarrow1^{-}$ e $x\rightarrow 1^{+}$.
\begin{equation}\frac{df_{1-}(x)}{dx}=\frac{df_{1+}(x)}{dx}\end{equation}
Eseguendo le derivate si ottiene:
\begin{equation}
\left\{
\begin{array}{l}
 -6x+H\;\; x \le 1\\ -2\frac{K}{x^3}\;\; x>1
\end{array}
\right.
\end{equation}
Calcolando le due derivate per $x=1$ si ottiene:
\begin{equation}\frac{df_{1-}(1)}{dx}=-6+H\;\;\;\frac{df_{1+}(1)}{dx}=-2K\end{equation}
In definitiva il sistema che ci permetterà di ottenere i parametri di $H$ e $K$ sarà:

\begin{equation}
\left\{
\begin{array}{l}
H-K=3\\ H-6=-2K
\end{array}
\right.
\end{equation}
Dopo alcuni semplici passaggi si arriva a calcolare i valori di $H$ e $K$:

\begin{equation}
\left\{
\begin{array}{l}
H=4\\K=1
\end{array}
\right.
\end{equation}

Dunque si avrà:

\begin{equation} f(x)=
\left\{
\begin{array}{l}
 -3x^2+4x\;\; x \le 1\\ \frac{1}{x^2}\;\; x>1
\end{array}
\right.
\end{equation}

La funzione risultante è illustrata nella seguente figura. Si osservi come nel punto di coordinate $(1,1)$ le due funzioni abbiano lo stesso valore e anche la stessa derivata. 
\begin{figure}[H]
\centering
\includegraphics[width=1\textwidth]{funzione-param.png}
\caption {$f(x)$ continua e derivabile in $x=1$}
\end{figure}

\item \textbf{Quesito 2}

Sia $f(x)$ una funzione reale di variabile reale, continua nel campo reale, tale che $f(0) = 2$. calcolare:
\begin{equation}\lim_{x\rightarrow 0}\frac{\int_{0}^x f(t)dt}{2xe^x}\end{equation}
\begin{equation}\int_{0}^x f(t)dt=F(x)-F(0)\end{equation}
\begin{equation}\lim_{x\rightarrow 0}\frac{F(x)-F(0)}{2xe^x}=\frac{F(0)-F(0)}{0}=\frac{0}{0}\end{equation}
Quindi il limite delle due funzioni genera una forma indeterminata del tipo $\frac{0}{0}$.

Verifichiamo che siano soddisfatte le condizioni di applicabilità del Teorema di De L'Hopital.

Il teorema di De L'Hopital, sotto opportune ipotesi, consente di calcolare il limite del rapporto tra due funzioni come il rapporto della derivata del numeratore e della derivata del denominatore.
\begin{equation}lim_{x \rightarrow a} \frac{h(x)}{g(x)}=lim_{x \rightarrow a} \frac{ \frac{dh(x)}{dx}}{\frac{dg(x)}{dx}}\end{equation}

Nel caso in esame si ha:

\begin{equation}h(x)=\int_{0}^x f(t)dt=F(x)-F(0)\end{equation}
\begin{equation}g(x)=2xe^x\end{equation}

\begin{equation}\frac{dh(x)}{dx}=\frac{d[F(x)-F(0)]}{dx}=f(x)\end{equation}
\begin{equation}f(0)=2\end{equation}
\begin{equation}\frac{dg(x)}{dx}=2e^x+2xe^x\end{equation}
\begin{equation}\frac{dg(x)}{dx}(x=0)=2e^0+0=2\end{equation}
Dunque si ha:
\begin{equation}\lim_{x\rightarrow 0}\frac{\int_{0}^x f(t)dt}{2xe^x}=\frac{2}{2}=1\end{equation}

\item \textbf{Quesito 3}

 Un protone in moto con velocità pari a $5\cdot10^6 m/s$ si muove in un campo magnetico uniforme uscente dalla pagina e di modulo $3\cdot10^{-3} T$; calcolare il raggio di curvatura della traiettoria descritta dal protone.

Un protone dotato di una certa velocità immerso in un campo magnetico è sottoposto ad una forza detta Forza di Lorentz il cui vettore risulta espresso dalla seguente relazione:
\begin{equation}\vec{F_L} = q \vec{v} {\times} \vec{B}\end{equation}


Il modulo di tale vettore $F_L$ risolta essere:
\begin{equation}F_L=qvBsin\alpha\end{equation}
dove $\alpha$ è l'angolo compreso tra il vettore campo magnetico $\vec{B}$ e il vettore velocità $\vec{v}$.

\begin{figure}[H]
\centering
\includegraphics[width=1\textwidth]{lorentz.jpg}
\caption {Forza di Lorentz}
\end{figure}

Nel caso in esame $\alpha=\ang{90}$. Dunque $sin(\alpha)=sin(\ang{90})=1$. Alla luce di ciò possiamo affermare che $F_L=qvB$.
La Forza di Lorentz è la sola forza applicata al protone. 

Per il secondo principio della dinamica possiamo dire che:
\begin{equation}F_L=ma_c\end{equation}
\begin{equation}a_c=\frac{v^2}{r}\end{equation}
\begin{equation}qvB=m\frac{v^2}{r}\end{equation}
Con semplici passaggi si ricava:
\begin{equation}r=\frac{mv^2}{qvB}=\frac{mv}{qB}\end{equation}
Sviluppando i calcoli si ricava:
\begin{equation}r=17,3\ m\end{equation}

\item \textbf{Quesito 4}

Sia dato il circuito in figura, con $R_1 = 5\ \Omega, R_2 = 4\ \Omega, R_3 = 10\ \Omega, R_4 = 50\ \Omega,  R_5 = 6\ \Omega, \Delta V=75\ V$. 

\begin{figure}[H]
\centering
\includegraphics[width=10cm]{circuito.png}
\caption {Circuito}
\end{figure}

Le due resistenze $R_2$ ed $R_3$ sono in parallelo. Dunque la loro resistenza equivalente risulta essere:
\begin{equation}R_{eq2,3}=\frac{R_2R_3}{R_2+R_3}=2.85\ \Omega\end{equation}
Analogamente le due resistenze $R_4$ ed $R_5$ danno una resistenza equivalente
\begin{equation}R_{eq4,5}=\frac{R_4R_5}{R_4+R_5}=5.3\ \Omega\end{equation}

\begin{figure}[H]
\centering
\includegraphics[width=10cm]{circuito-equiv.jpg}
\caption {Circuito equivalente}
\end{figure}

La resistenza equivalente di tutto il circuito è dunque:
\begin{equation}R_{eq}=R_1+R_{eq2,3}+R_{eq4,5}=13.2\ \Omega\end{equation}
Per la legge di Ohm possiamo quindi calcolare la corrente $I_{tot}$ che scorre nel circuito come:
\begin{equation}I_{tot}=\frac{\Delta V}{R_{eq}}=5,68A\end{equation}

Una volta nota la corrente $I_{tot}$ possiamo calcolare la differenza di potenziale $\Delta V_{2,3}$ ai capi della resistenza $R_{eq2,3}$ e la differenza di potenziale $\Delta V_{4,5}$ ai capi della resistenza $R_{eq4,5}$

\begin{equation}
\left\{
\begin{array}{l}
\Delta V_{2,3}=R_{eq2,3}I_{tot}=16.18\ V\\ \Delta V_{4,5}=R_{eq4,5}I_{tot}=30.10\ V
\end{array}
\right.
\end{equation}

Una volta ricatave le differenze di potenziale ai capi delle resistenze equivalenti $R_{eq2,3}$ e $R_{eq4,5}$ possiamo calcolare le correnti che attraversano le varie resistenze:
\begin{equation}
\left\{
\begin{array}{l}
I_{R_2}=\frac{\Delta V_{2,3}}{R_2}=4.04\ A\\ I_{R_3}=\frac{\Delta V_{2,3}}{R_3}=1.62\ A \\ I_{R_4}=\frac{\Delta V_{4,5}}{R_4}=0.6\ A \\ I_{R_5}=\frac{\Delta V_{4,5}}{R_5}=5.01\ A
\end{array}
\right.
\end{equation}
Infine le differenze di potenziale ai capi delle resistenze risultano essere:
\begin{equation}
\left\{
\begin{array}{l}
\Delta V_{R2}=\Delta V_{R3}= \Delta V_{2,3}=16.18\ V \\
\Delta V_{R4}=\Delta V_{R5}= \Delta V_{4,5}=30.10\ V
\end{array}
\right.
\end{equation}

Inoltre si è provato ad usare un simulatore circuitale online (Circuitlab) per verificare i risultati ottenuti.
\begin{figure}[H]
\centering
\includegraphics[width=1\textwidth]{circuitlab.jpg}
\caption {Circuito realizzato con Circuitlab}
\end{figure}

A meno di qualche errore di arrotondamento i risultati numerici coincidono con quelli forniti dal simulatore. 

Si osservi che la differenza di potenziale ai capi del parallelo $R_{eq2,3}$ si calcola come la differenza tra $V(out1)=46.62\ V$ e $V(out2)=30.41\ V$ che coincide con $\Delta V_{eq2,3}=16.18\ V$.

Come si può vedere dai risultati si evince che nel parallelo di due resistenze, la corrente $I_{tot}$ passi maggiormente nella resistenza di valore minore a parità di differenza di potenziale applicata ai capi delle due resistenze in parallelo. Questo vale sia per  $R_{eq2,3}$ che per $R_{eq4,5}$.
\end{enumerate}

\section{Riferimenti}
Per la realizzazione dell'elaborato ci si è avvalsi delle seguenti risorse disponibili sul Web.

\begin{enumerate}
\item \textbf{Miktek}, https://miktex.org/, \textit{Programma di editing per Latex}
\item \textbf{Tutorial Miktek}, http://www.docs.is.ed.ac.uk/skills/documents/3722/3722-2014.pdf/, \textit{Introduzione all'uso di Latex con numerosi esempi}
\item \textbf{Corso  Latex}, https://users.dimi.uniud.it/~gianluca.gorni/TeX/itTeXdoc/CorsoTeX.pdf, \textit{Corso avanzato di Latex}
\item \textbf{Desmos}, https://www.desmos.com/calculator/, \textit{Programma per la visualizzazione di funzioni}
\item \textbf{Schematic Entry}, https://www.digikey.com/schemeit/, \textit{Programma per disegnare circuiti}
\item \textbf{Circuitlab}, https://www.circuitlab.com/, \textit{Simulatore circuitale online}

\end{enumerate}

\end{document}