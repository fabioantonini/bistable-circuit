% psccat.tex		 pst-circ catalog of symbols
% Nov 18, 1999
%%%%%%%%%%%%%%%%%%%%%%%%%%%%%%%%%%%%%%%%%%%%


\documentclass[12pt,a4paper]{article}
%\usepackage{pst-circ}
%\usepackage{pscirc3}
%\usepackage{fancyvrb,fvrb-ex}
\usepackage{circuitikz,siunitx}
\usepackage[figure,tworuled,linesnumbered,noline]{algorithm2e}
\usepackage[italian]{babel}
\usepackage[latin1]{inputenc}
\addtolength{\textwidth}{2cm}
\addtolength{\oddsidemargin}{-1cm}
\setlength{\parindent}{0pt}

\title{\vspace*{-2cm}
\texttt{circuitikz}: alcuni esempi}
\author{G. Antonini}
\date{December 29, 2021}

%%%%%%%%%%%%%%%

\begin{document}
\maketitle
%%%%%%%%%%%%%%%%%%%%%%%%%%%%%%%%%%%%%%%%%%%%

\section{Circuito RC serie}
%
\begin{figure}[!ht]
\begin{center}
\begin{circuitikz}
  \draw(0,0)
  to[V,v=$v_s(t)$](0,2)
  %to[short](3,2)
  to[R=$R$](3,2)
  to[C=$C$](3,0)
  to[short](0,0);
\end{circuitikz}
\caption{\small Circuito RC serie.} \label{fig:RC_serie}
\end{center}
\end{figure}
%
\section{Circuito RL parallelo}
%
\begin{figure}[!ht]
\begin{center}
\begin{circuitikz}[american, voltage shift=2]
  \draw (0,0) to[isource, l=$I_{SC}$] (0,3)
  to[short, -*, i=$I_0$] (2,3)
  to[R=$G_{eq}S$, i>_=$i_R$] (2,0) -- (0,0);
  \draw (2,3) -- (4,3)
  to[L,l_=$L$, i>_=$i_L$, v^=$v_L$]
  (4,0) to[short, -*] (2,0);
\end{circuitikz}
\caption{\small Circuito RL parallelo.} \label{fig:RL_parallelo}
\end{center}
\end{figure}
%

\section{Circuito RL parallelo capovolto}
%
\begin{figure}[!ht]
\begin{center}
\begin{circuitikz}[american, voltage shift=1]
  \draw (0,3) to[isource, l_=$I_{SC}$] (0,0)
  to[short, -*, ] (2,0)
  to[R=$G_{eq}S$, i>^=$i_R$] (2,3) -- (0,3);
  \draw (2,0) -- (4,0)
  to[L=$L$, i>^=$i_L$, v=$v_L$]
  (4,3) to[short, -*] (2,3);
\end{circuitikz}
\caption{\small Circuito RL parallelo.} \label{fig:RL_parallelo}
\end{center}
\end{figure}
%
%\clearpage
%
\section{Opamp}
%
\begin{figure}[!ht]
\begin{center}
\begin{circuitikz}[american, voltage shift=1]
\ctikzset{bipoles/length=1cm}
\draw
(0, 0) node[op amp] (opamp) {}
(opamp.-) -- (-3, 0.35) to [V=$v_1(t)$, invert] (-3,-1) to (-3,-1.1) to[L=$L$, i>^=$i_L$, v=$v_L$] (-3,-2.5) node[ground]{}
(opamp.down) node[ground] {}
(opamp.-) to[short,*-] ++(0,0.5) coordinate (leftC)
to[R=$R_f$] (leftC -| opamp.out)
to[short,-*] (opamp.out) to [short,-o] (1.5,0) %to (1.5,-0.5) node[ground]{}
(opamp.+) -- (-0.85,-1.5) to[R=$R_B$] (-0.85,-2.5) -- (-0.85,-3) node[ground]{}
(-0.85,-1.2) to[R,l_=$R_A$] (0.85,-1.2) to[short,*-] (0.85,0)%(leftC -| opamp.out)
;
\end{circuitikz}
%\begin{circuitikz}[scale=0.8, transform shape]
% \draw (0,0) node[above]{$v_i$} to[short, o-] ++(1,0)
% node[op amp, anchor=-](OA){\texttt{OA1}}
% (OA.+) -- ++(0,-1) coordinate(FB)
% to[R=$R_1$] ++(0,-2) node[ground]{}
% (OA.-) -- ++(0,+1) coordinate(FA)
% (FA) to[R=$R_2$, *-] (FA -| OA.out) -- (OA.out)
% to [short, *-o] ++(1,0) node[above]{$v_o$}
% ;
%\end{circuitikz}
\caption{\small Bistabile.} \label{fig:bistabile}
\end{center}
\end{figure}

\end{document}
